%%%%%%%%%%%%%%%%%%%%%%%%%%%%%%%%%%%%%%%%%
% Twenty Seconds Resume/CV
% LaTeX Template
% Version 1.0 (14/7/16)
%
% Original author:
% Carmine Spagnuolo (cspagnuolo@unisa.it) with major modifications by 
% Vel (vel@LaTeXTemplates.com) and Harsh (harsh.gadgil@gmail.com)
%
% License:
% The MIT License (see included LICENSE file)
%
%%%%%%%%%%%%%%%%%%%%%%%%%%%%%%%%%%%%%%%%%

\documentclass[letterpaper]{twentysecondcv} % a4paper for A4

\usepackage[spanish]{babel}

% Burbujas de habilidades
\newcommand\skills{ 
~
    \smartdiagram[bubble diagram]{
        \textbf{Desarrollo}\\\textbf{de Software},
        \textbf{Diseño}\\\textbf{Web Front-end},
        \textbf{Aplicaciones}\\\textbf{Móviles},
        \textbf{Ciencia}\\\textbf{de Datos},
        \textbf{~~~Sistemas~~~}\\\textbf{~~Embebidos~~},
        \textbf{Docencia}
    }
}

% Barras de conocimiento de lenguajes
\programming{ {PHP \textbullet Java \textbullet SQL / 2.5}, {OpenSCAD \textbullet Matlab / 5}, {C++ \textbullet Python \textbullet JS \textbullet HTML5 \textbullet CSS3 / 5.5}}

% Cursos realizados
\courses{
    \textbullet Análisis visual de grandes volúmenes de datos. [90 hs.] \\
    \textbullet Sistemas Colaborativos con Restricciones Temporales. [60 hs.] \\
    \textbullet Minería de Datos y Aprendizaje Automatizado. [90 hs.] \\
    \textbullet Tópicos en Big Data. [80 hs.]\\
    \textbullet Modelos Matemáticos de Simulación en la Inv. Agrop. [60 hs.]\\
    \textbullet Sist. Distribuidos de Tiempo Real. [60 hs.]\\
    \textbullet IoT y el Control, un Enfoque Holístico. [60 hs.]\\
}


%----------------------------------------------------------------------------------------
%    INFORMACION PERSONAL
%----------------------------------------------------------------------------------------

\cvname{MATÍAS MICHELETTO}
\cvjobtitle{ Ingeniero Electrónico }
\cvlinkedin{in/matiasmicheletto}
\cvgithub{matiasmicheletto}
\cvnumberphone{+54 9 291 459 5181}
\cvsite{matiasmicheletto.github.io}
\cvmail{matias.micheletto@uns.edu.ar}
\cvfacebook{/miche1989}
\cvinstagram{/matias.jm_}
\cvyoutube{/channel/UCI0mjA0Hl7jok8DzrWavTng}

%----------------------------------------------------------------------------------------

\begin{document}

\makeprofilefirst % Sidebar de primera pagina

\section{Resumen}
\begin{twenty}

Acabo de defender mi tesis para aspirar al título de Doctor en Ingeniería y actualmente \\
me desempeño como docente de las asignaturas de grado Técnicas Digitales y Diseño  \\
de Circuitos Lógicos, en la Universidad Nacional del Sur.\\

Durante el periodo doctoral, en paralelo al cursado de materias de posgrado y a las \\
actividades de investigación, he adquirido conocimiento y experiencia en desarrollo \\
de aplicaciones web, de escritorio, móviles y en implementación de sistemas \\
embebidos.

\end{twenty}

\section{Educación}
\begin{twenty}
    %\twentyitem{<dates>}{<title>}{<organization>}{<location>}{<description>}
    \twentyitem
        {2020}
        {}
        {Doctorado en Ingeniería.}
        {Calificación: Sobresaliente (10/10)}
        {\href{http://www.uns.edu.ar/}{Universidad Nacional del Sur.}}
        {}
    \twentyitem
        {2016}
        {}
        {Ingeniería Electrónica.}
        {Promedio: 8.17/10.0}
        {\href{http://www.uns.edu.ar/}{Universidad Nacional del Sur.}}
        {}
    \twentyitem
        {2006}
        {}
        {Cs. Naturales y Téc. en Prod. Agropecuaria.}
        {Promedio: 7.79/10.0}
        {Escuela de Agricultura y Ganadería U.N.S.}
        {}
\end{twenty}

\section{Docencia}
\begin{twenty}
    \twentyitem
        {2018 - Pres.}
        {}
        {Asistente de cátedra dedicación simple}
        {\href{http://www.diec.uns.edu.ar/}{DIEC - UNS}}
        {}
        {Diseño de Circuitos Lógicos - Técnicas Digitales}
    \twentyitem
        {2016 - 2018}
        {}
        {Ayudante de cátedra graduado}
        {\href{http://www.diec.uns.edu.ar/}{DIEC - UNS}}
        {}
        {Diseño de Circuitos Lógicos - Técnicas Digitales}
\end{twenty}

\section{Experiencia}
\begin{twenty} 
\twentyitem
        {2019 - 2020}
        {}
        {Desarrollo front-end}
        {\href{http://www.webcapp.com/}{CAPP (CAPP Mobile S.R.L.)}}
        {}
        {\begin{itemize}
        \item Colaboración en el desarrollo de la app móvil, backoffice, panel administrador y tienda online de CAPP.
        \end{itemize}}\\
\twentyitem
        {2017 - 2019}
        {}
        {Desarrollo front-end e IoT}
        {\href{http://www.cinaweb.org/}{CINA (Neufitech S.R.L.)}}
        {}
        {\begin{itemize}
        \item Desarrollo de plataforma web para aplicaciones de actividades neuro-psicológicas (PsiMESH).
        \item Colaboración en desarrollo y mantenimiento del sistema de gestión de turnos de CINA.
        \item Diseño e implementación de dispositivos IoT para asistencia en comunicación y movilidad de personas con discapacidades motrices.
        \end{itemize}}
\end{twenty}

\vspace{2mm}

\section{Pasantías}
\begin{twenty}
    \twentyitem
        {2014 - 2015}
        {}
        {Pasantía Interna}
        {DIEC-UNS}
        {\textit{``Desarrollo de un Libro Multimedial de Mecánica''.}}
        {\textbf{Director:} Dr. Gustavo Gasaneo. \textbf{Co-Director:} Dr. Claudio Delrieux.} \\
\end{twenty}

\section{Premios}
\begin{twenty}
    \twentyitem
        {2018}
        {}
        {Agroton 2018}
        {Segundo premio}
        {\textbf{Equipo A:} Matías Micheletto, Alejandro André, Matías Timi, José Augusto Strick, Franco Tronelli y Guido Temperini.}
        {\textbf{Propuesta:} Paquete de Automatización e Información Integral de Soluciones Agropecuarias (PAIISA)}
\end{twenty}


\end{document} 
