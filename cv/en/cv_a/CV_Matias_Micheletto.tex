\documentclass[10pt]{article}

\usepackage[left=1in, right=1in, bottom=2cm]{geometry}
\usepackage[spanish]{babel}
\usepackage[utf8]{inputenc}
\usepackage{array, xcolor}
\usepackage{censor}
\usepackage[urlbordercolor={1 1 1}]{hyperref}
\usepackage{longtable}

\newcolumntype{L}{>{\raggedleft}p{0.14\textwidth}}
\newcolumntype{R}{p{0.8\textwidth}}
\newcommand\VRule{\color{lightgray}\vrule width 0.5pt}


%\def\censorData{1} % Censurar datos personales para publicar


\title{\bfseries\huge Resume}
\author{\Large Matías J. Micheletto}
\date{July 2022}

\begin{document}
\maketitle

\section{Personal data}
\raggedright
\textbf{Full name:} Micheletto, Matías Javier \\
%\textbf{National ID Number:}
%\ifx\censorData\undefined
%  34.377.688 \\
%\else
%	\censor{34.377.688} \\
%\fi
%\textbf{Place and date of birth:}
%\ifx\censorData\undefined
%  	Bahía Blanca, april 26 1989 \\
%\else
%	\censor{Bahía Blanca, april 26 } 1989 \\
%\fi
\textbf{Nacionality:} Argentine  \\
\textbf{Civil status:} Single \\
\textbf{Work address:}
\ifx\censorData\undefined
  	CIT Golfo San Jorge, Ruta 1 KM 4, Comodoro Rivadavia (9000), Chubut. \\
\else
	\censor{CIT Golfo San Jorge, Ruta 1 KM 4, Comodoro Rivadavia (9000), Chubut.} \\
\fi
\textbf{Phone:}
\ifx\censorData\undefined
  	+54 9 xxx xxx xxxx \\
\else
	\censor{ +54 9 291 459 5181 - \textbf{Fax:} +54 9 291 459 5154 } \\
\fi
\textbf{Email:} matias.micheletto@uns.edu.ar \\

\section{About}
After obtainng my Electrical Engineering degree and during my doctoral stance, I have acquired experience and skills in technological solutions development applied to a wide range of fields, from embedded systems development, through scientific computing methods, to mathematical modeling and multiplatform software development. I have worked in both professional and academic environments, working independently or in multidisciplinary teams. I have published articles in international journals, won innovation competitions and participated in many national and international conferences.

\section{Education}
\begin{tabular}{L!{\VRule}R}
2021 & {\bf National University of the South}\\
	 & PhD Engineering. \\
	 & Thesis title: ``Development of Computer Systems for Supporting Agricultural Work in the Southwest of the Buenos Aires Province'' \\
	 & Thesis grade: 10/10. \\[5pt]

2016 & {\bf National University of the South}\\
	 & Electronic Engineer. \\
	 & Thesis title: ``Scheduling of Optional-Mandatory Real-Time Tasks in Homogeneous Multicore Systems with Energy Constraints'' \\
	 & GPA: 8.17/10. \\[5pt]

2006 & {\bf Escuela de Agricultura y Ganadería U.N.S.} \\
 	 & Agriculture Production B.S.\\
	 & GPA: 7.79/10. \\
\end{tabular}

\section{Teaching experience}
\begin{tabular}{L!{\VRule}R}
2018 - 2020 & {\bf Teaching assistant} \\
	 & Digital Electronics and Logic Circuit Design. \\[5pt]
	 
2016 - 2018 & {\bf Graduate teaching collaborator} \\
	 & Digital Electronics and Logic Circuit Design. \\
\end{tabular}

\section{Scholarships}
\begin{longtable}{L!{\VRule}R}
2021 - 2024 & {\bf CONICET Postdoctorate Intern Scholarship} \\
	& \textit{``Automated Learning for Smart Decision Making in Precision Agriculture in the South West of Buenos Aires Province''.} \\
   	& \textbf{Advisor:} Dr. Carlos Chesñevar. \\
   	& \textbf{Co-Advisor:} Dr. Juan Galantini. \\ [5pt]

2016 - 2021 & {\bf CONICET PhD Intern Scholarship} \\
 	& \textit{``Development of Computer Systems for Supporting Agricultural Work in the Southwest of the Buenos Aires Province''.} \\
	& \textbf{Advisor:} Dr. Rodrigo Santos. \\
	& \textbf{Co-Advisor:} Dr. Juan Galantini. \\ [5pt]

2015 - 2016 & {\bf PGI-MAyDS Research Scholarship} \\
	& \textit{``Sustainability in Civil Infrastructure''.} \\
	& \textbf{Advisor:} Dr. Néstor Ortega. \\
	& Secretaría General de Ciencia y Tecnología U.N.S. Res. CSU-752/15. \\[5pt]

2014 & {\bf Introduction to Scientific Research for Advanced Students} \\
	& \textbf{Tema:}\textit{``Optimal Real-Time Scheduling in Multicore Systems''.} \\
	& \textbf{Advisor:} Dr. Javier Orozco. \\
	& Secretaría General de Ciencia y Tecnología U.N.S. \\
	& Res. CSU-135/14. \\[5pt]

2014 & {\bf Intern Scholarship Incentive Program} \\
	& Dpto. de Ingeniería Eléctrica y Computadoras U.N.S. \\
	& Res. CSU-213/14. \\[5pt]

2013 & {\bf Intern Scholarship Incentive Program} \\
	& Dpto. de Ingeniería Eléctrica y Computadoras U.N.S. \\
	& Res. CSU-128/13. \\
\end{longtable}

\section{Journal articles}
\begin{longtable}{L!{\VRule}R}
2022 & {\bf Flow Scheduling in Data Center Networks with Time and Energy Constraints: A Software-Defined Network Approach.} \\
	& Martin Fraga, Matías Micheletto, Andrés Llinás, Rodrigo Santos and Paula Zabala. \\
	& Future Internet | MDPI. \\[5pt]

2022 & {\bf A novel approach for sEMG gesture recognition using resource-constrained hardware platforms.} \\
	& Matías Micheletto, Carlos Chesñevar and Rodrigo Santos. \\
	& Computer and Information Systems. \\[5pt]

2021 & {\bf An IoT-based Infraestructure to Enhance Self-Evacuations in Natural Hazardous Events.} \\
 	 & Mariano Finochietto, Gabriel Eggly, Matías Micheletto, Roger Pueyo Centelles, Roc Messeguer, Sergio Ochoa, Rodrigo Santos, Javier Orozco. \\
 	 & Personal ans Ubiquitous Computing. \\[5pt]
\newpage
2020 & {\bf Development and Validation of a LiDAR Scanner for 3D Evaluation of Soil Vegetal Coverage.} \\
	 & Matías Micheletto, Luciano Zubiaga, Rodrigo Santos, Juan Galantini, Miguel Cantamutto, Javier Orozco. \\
	 & Electronics, 9(1), 109. \\[5pt]
	 
2019 & {\bf Utilizando UML para el aprendizaje del modelado y diseño de sistemas ciber-físicos.} \\
	 & Leonardo Ordínez, Rodrigo Santos, Gabriel Eggly, Matías Micheletto. \\
	 & IEEE-RITA, 15(1), 50-60. \\[5pt]

2019 & {\bf Scheduling Mandatory-Optional-Time Tasks in Homogeneous Multi-Core Systems with Energy Constraints Using Bio-Inspired Meta-Heuristics.} \\
	 & Matías Micheletto, Rodrigo Santos, Javier Orozco. \\
	 & Journal of Universal Computer Science, 25(4), 390-417. \\[5pt]

2018 & {\bf Flying Real-Time Network to Coordinate Disaster Relief Activities in Urban Areas.} \\
	 & Matías Micheletto, Vinicius Petrucci, Rodrigo Santos, Javier Orozco, Daniel Mosse, Sergio Ochoa, Roc Meseguer. \\
	 & Sensors, 18(5), 1662. \\[5pt]
	 
2017 & {\bf Real-Time Communication Support for Underwater Acoustic Sensor Networks.} \\
	 & Rodrigo Santos, Javier Orozco, Matías Micheletto, Sergio Ochoa, Roc Meseguer, Pere Milan, Carlos Molina. \\
	 & Sensors, 17(7), 1629. \\	 
\end{longtable}

\section{Awards}
\begin{tabular}{L!{\VRule}R}
2021 & {\bf Agrotón 2021 - 24h innovation contest} \\
	& \textbf{First Place.} \\
	& \textbf{Equipo A:} Matías Micheletto, Alejandro André, Matías Timi, Martín Rodríguez and Guido Temperini. \\
	& \textbf{Technological proposal:} Traceability of Meat Products Based on Agrigenomics, DNA-Hashing and Blockchain (TR4). \\

2018 & {\bf Agrotón 2018 - 24h innovation contest} \\
	 & \textbf{Second Place.} \\
	 & \textbf{Equipo A:} Matías Micheletto, Alejandro André, Matías Timi, José Augusto Strick, Franco Tronelli and Guido Temperini. \\
	 & \textbf{Technological proposal:} Automation and Integral Information Package for Agriculture Solutions (PAIISA). \\
\end{tabular}

\section{Posgraduate courses}
\begin{longtable}{L!{\VRule}R}
2020 & {\bf ``Advanced Techniques for Evolutionary Computation''.} \\
	& Department for Computer Science and Engineering \\
	& By: Dr. Ingnacio Ponzoni. \\
	& Duration: 120hs.  Grade: 10/10. \\[5pt]
\newpage 
2020 & {\bf ``Visual Analysis for Big Data''.} \\
   & Department for Computer Science and Engineering \\
   & By: Dr. Silvia Castro and Dr. Ma. Luján Ganuza. \\
   & Duration: 90hs.  Grade: 10/10. \\[5pt]

2020 & {\bf ``Time Aware Collaborative Systems''.} \\
   & Department for Electrical Engineering and Computers \\
   & By: Dr. Javier Orozco and Dr. Rodrigo Santos. \\
   & Duration: 60hs.  Grade: 10/10. \\[5pt]

2017 & {\bf ``Datamining and Machine Learning''.} \\
   & Department for Computer Science and Engineering \\
   & By: Dr. Carlos Chesñevar. \\
   & Duration: 90hs.  Grade: 10/10. \\[5pt]

2017 & {\bf ``Introduction to Big Data''.} \\
   & Department for Electrical Engineering and Computers \\
   & By: Dr. Claudio Delrieux. \\
   & Duration: 80hs. Grade: 10/10. \\[5pt]

2017 & {\bf ``Mathematical Models and Simulation in Agriculture Research''.} \\
   & Departamento de Agronomía. \\
   & By: Dr. Juan Galantini. \\
   & Duration: 60hs.  Grade: 10/10. \\[5pt]

2016 & {\bf ``Real Time Distributed Systems''.} \\
   & Department for Electrical Engineering and Computers \\
   & By: Dr. Ricardo Cayssials and Dr. Edgardo Ferro. \\
   & Duration: 60hs. Grade: 10/10. \\[5pt]

2016 & {\bf ``Internet of Things and Control, an Holistic Approach''.} \\
   & Department for Electrical Engineering and Computers \\
   & By: Dr. Rodrigo Santos and Dr. Sergio Ochoa. \\
   & Duration: 60hs. Grade: 10/10. \\[5pt]
\end{longtable}

\section{Internships}
\begin{tabular}{L!{\VRule}R}
2014 & {\bf Department for Electrical Engineering and Computers - U.N.S. Internship} \\
	 & Multimedia Interactive Applications for Physics Education. \\
	 & Duration: 10 month. \\
\end{tabular}

\section{Conferences}
\begin{longtable}{L!{\VRule}R}
2021 & {\bf 50º Jornadas Argentinas de Informática} \\
	& Congreso de AgroInformática (CAI) \\
	& \textit{``Desarrollo de una aplicación híbrida para cálculos de siembra''.} \\
	& Matías J. Micheletto, Gabriel M. Eggly, Juan P. D'Amico and Santiago J. Crocioni. \\[5pt]

2021 & {\bf 50º Jornadas Argentinas de Informática} \\
	& Simposio Argentino de Informática Industrial e Investigación Operativa (SIIO) \\
	& \textit{``Planificación de flujos con vencimiento en redes definidas por software''.} \\
	& Martín Fraga, Andres Llinás, Matías J. Micheletto, Rodrigo M. Santos and Paula Zabala. \\[5pt]
\newpage
2020 & {\bf International Conference on Computer Science and Intelligent Systems} \\
	& \textit{``Unattended Crowdsensing Method to Monitor the Quality Condition of Dirt Roads''.} \\
	& Matías J. Micheletto, Rodrigo Santos, Sergio Ochoa. \\[5pt]

2020 & {\bf 49º Jornadas Argentinas de Informática} \\
	& Simposio Argentino de Educación en Informática (SAEI) \\
	& \textit{``Cipressus: Un Sistema de Gestión de Contenido para el Aprendizaje de Sistemas Digitales''.} \\
	& Matías J. Micheletto. \\[5pt]

2019 & {\bf 13$^{th}$ International Conference on Ubiquitous Computing \& Ambient Intelligence} \\
	 & \textit{``Evacuation Supporting System Network (ESSN) based on IoT Components''.} \\
	 & Gabriel Eggly, José Mariano Finochietto, Matías Micheletto, Rodrigo Santos, Sergio Ochoa, Roc Meseguer and Javier Orozco. \\[5pt]

2019 & {\bf 48º Jornadas Argentinas de Informática} \\
	 & 1º Taller Argentino de Internet de las Cosas (TAIC) \\
	 & \textit{``Internet de las Cosas como Bien Social''.} \\
	 & Gabriel Eggly, Mariano Finochietto, Matías J. Micheletto, Rodrigo Santos. \\[5pt]

2018 & {\bf 47º Jornadas Argentinas de Informática} \\
	 & 10º Congreso de AgroInformática (CAI 2018) \\
	 & \textit{``Diseño e Implementación de un Escáner Lidar para Análisis Tridimensional de Covertura Vegetal de Suelos''.} \\
	 & Matías J. Micheletto, Rodrigo Santos, Luciano Zubiaga, Juan Galantini. \\[5pt]

2017 & {\bf XXV Jornadas de Jóvenes Investigadores} \\
	 & \textit{``Planificación Óptima de un Sistema Multiprocesador de Tiempo Real con Restricciones de Precedencia, Comunicación y Energía''.} \\
	 & Matías J. Micheletto, Alejandro Borghero, Gabriel M. Eggly. \\[5pt]

2017 & {\bf 46º Jornadas Argentinas de Informática} \\
	 & 9º Congreso de AgroInformática (CAI 2017) \\
	 & \textit{``Desarrollo de una Aplicación Móvil para Cálculos de Pulverizaciones Agrícolas''.} \\
	 & Gabriel M. Eggly, Matías J. Micheletto, Juan P. D'Amico, Santiago J. Crocioni. \\[5pt]

2016 & {\bf V Congreso Internacional sobre Cambio Climático y Desarrollo Sustentable} \\
	 & \textit{``Afectación de la Luna en la Medición de la Contaminación Lumínica''.} \\
	 & Luciana C. Lambertucci, Matías J. Micheletto, Jorge A. Starobinsky, Néstor F. Ortega. \\[5pt]

2016 & {\bf 10$^{th}$ International Conference on Ubiquitous Computing \& Ambient Intelligence} \\
	 & \textit{``Scheduling Real-Time Traffic in Underwater Acoustic Wireless Sensor Networks''.} \\
	 & Rodrigo Santos, Javier Orozco, Matías Micheletto, Sergio Ochoa, Roc Meseguer, Pere Millan, Carlos Molina. \\[5pt]

2016 & {\bf 45º Jornadas Argentinas de Informática} \\
	 & 8º Congreso de AgroInformática (CAI 2016) \\
	 & \textit{``Diseño e Implementación de un Registrador de Esfuerzos para Maquinaria Agrícola''.} \\
	 & Matías J. Micheletto, Gabriel M. Eggly, Rodrigo Santos. \\[5pt]

2015 & {\bf V Brazilian Symposium on Computing Systems Engineering} \\
	 & \textit{``Using bioinspired meta-heuristics to solve reward-based energy-aware mandatory/optional real-time tasks scheduling''.} \\
	 & Matías J. Micheletto, Javier D. Orozco, Rodrigo Santos.  \\[5pt]
\newpage
2014 & {\bf Simposio Argentino de Sistemas Embebidos} \\
	 & Foro tecnológico 2014 \\
	 & \textit{``Design and Implementation of an Embedded Prototype for Monitoring a Combine Harvester''.} \\
	 & Ana S. Arauz Lozano, Matías J. Micheletto, Leonardo Ordinez, Rodrigo Santos.  \\[5pt]

2012 & {\bf 41º Jornadas Argentinas de Informática} \\
	 & Concurso de Trabajos Estudiantiles U.N.L.P. \\
	 & \textit{``Optimizador de Funciones Multivariadas por Enjambre de Partículas''.} \\
	 & Matías J. Micheletto. \\
\end{longtable}

\section{Project evaluation}
\begin{tabular}{L!{\VRule}R}
2019 & {\bf FONCyT-ANPCyT} \\
	 & PICT-START UP-2018 \\
	 & Collaboration as Project Evaluator Peer. \\[5pt]

2017 & {\bf CLEI 2017 / 46 JAIIO} \\
	 & Simposio Latinoamericano de Infraestructura, Hardware y Software 2017. \\
	 & Collaboration as Article Reviewer. \\
\end{tabular}

\section{Language}
\begin{tabular}{L!{\VRule}R}
    & {\bf Spanish} \\
    &  Native speaker. \\[5pt]
    
    & {\bf English} \\
    & B2-Upper-Intermediate (Listening: B1-Intermediate, Reading: C2-Proficient). \\[5pt]
\end{tabular}

\end{document}
\grid
